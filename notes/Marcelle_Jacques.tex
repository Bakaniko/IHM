\subsubsection{Marcelle et Jacques, clients
réguliers}\label{marcelle-et-jacques-clients-ruxe9guliers}

Marcelle et Jacques sont des clients réguliers de l'Opéra. Tous les pour
Noël et leur anniversaire de mariage, ils réservent un billet pour un
opéra.

En vue de leur anniversaire de mariage dans quelques mois, en mai, ils
se rendent sur le site de l'Opéra national de Paris.

Ils cliquent sur le lien ``Programmation 2017'' qui les redirigent vers
la programmation de cette année. Habitués du site, ils utilisent le
formulaire et choisissent un spectacle de type Opéra,Jacques n'aime pas
les ballets, et le mois de mai à l'aide des listes déroulantes. Puis ils
cliquent sur ``Rechercher''.

Le site leur propose alors, ``Les contes d'Hoffmann'' le 5 mai et
``Aida''le 13 mai.

Ils cliquent sur les ``Contes d'Hoffmann'' et sont redirigés vers la
page du spectacle. Après l'avoir consulté quelques instants, ils
reviennent en arrière pour consulter la page d'``Aida''. Préférant
Verdi, ils choisissent cette représentation en cliquant sur le bouton
``Réserver'' placé à côté de la date qui les intéresse.

Habitués de l'Opéra national, ils choisissent deux places biens placées
dans la fosse d'orchestre. Après avoir ajouté les deux places dans leur
panier, ils cliquent sur le bouton ``Étape suivante''. Ils sont alors
redirigés vers le panier qu'ils valident en cliquant sur ``Paiement''.

Il leur est alors demandé de se connecter. Après avoir saisi leurs
identifiants, ils sont renvoyés vers la page de paiement. Après avoir
réglé la transaction, ils sont redirigés vers leur compte, où ils
peuvent imprimer leurs billets.
