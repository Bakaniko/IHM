\subsubsection{Philippe, administrateur}\label{philippe-administrateur}

Philippe est un des administrateurs de l'Opéra National de Paris. C'est
lui qui ajoute de nouveaux spectacles et plannifie les représentations
en accord avec le Directeur. Il les créé et les modifie si besoin. Il
peut supprimer les objets inutiles ou erronés.

Ce matin, par exemple, il a remarqué que l'affiche des contes d'Hoffman
n'est pas la bonne. Il se connecte donc sur le site. Il apprécie le fait
que son interface soit la même que celle des clients, ainsi il peut voir
régulièrement si il y a des soucis.

Il se rend donc sur le site et se rend sur la page de connexion. Le site
reconnait que Philippe est un administrateur et le redirige
automatiquement vers la page de gestion.

SUr cette page, Philippe a la possibilité d'ajouter, modifier ou
supprimer un spectacle. Il peut aussi administrer de la même façon les
salles, réservations et utilisateurs.

En cliquant sur ``Gestion des spectacles'', il accède à la page
permettant la gestion de ces derniers. Sur le côté, il peut choisir dans
la liste tous les spectacles présents dans la base. Il sélectionne ``Les
contes d'Hoffmann'' et valide.

La fiche du spectacle apparaît. En regardant le champ contenant le nom
de l'image, il s'aperçoit qu'elle porte une extension peu courante:
``.webp''. Il vérifie le fichier sur le serveur et s'aperçioit que
l'extension est en ``.jpg''. Il change donc l'extension dans le
formulaire et valide. Puis il retourne sur l'accueil et contrôle le
changement sur la fiche du spectacle. L'affiche offcielle est maintenant
affichée.
