\subsubsection{Jean-Marc, prospect}\label{jean-marc-prospect}

Jean-Marc n'est pas encore client de l'Opéra.

Mais il souhaite assister à une représentation de Rigoletto dont il a vu
les affiches dans la rue.

Le soir en rentrant du travail, il accède au site de l'opéra. La page
d'acceuil ne présente pas Rigoletto, il clique donc sur Programmation
2017 et est redirigé vers la page de programmation.

Il voit dans la liste des représentations en dessous du formulaire
qu'une représentation de Rigoletto est prévue pour décembre. Il clique
donc sur le lien qui le renvoit vers la description du spectacle.

Il peut admirer l'affiche, apprendre le nom du metteur en scène et des
artistes. Un menu déroulant lui permet de choisir une représentation. Il
s'aperçoit qu'il n'avait pas vu celle d'octobre auparavant. Il choisit
celle-ci, plus proche dans le temps.

Puis il clique sur sur le bouton ``Réserver'', et est redirigé vers la
page contenant le plan de salle. Il voit que les sièges les plus chers
sont devant mais en dehors de ces moyens. Il sélectionne donc un siège
libre dans le parterre.

Il constate que sa sélection apparaît dans le panier.

Ne souhaitant pas faire d'autres réservations, il se rend sur son
panier. Sa place y est bien présente.

il clique donc sur le bouton ``Paiement''. Le site lui demande alors de
créer un compte.

Il saisie ses identifiants habituels ainsi que son adresse mail et
valide. Il se rend sur sa boîte mail pour attendre le mail de
confirmation. L'ayant reçu, il l'ouvre et clique sur le lien.

Il est alors redirigé vers son panier. Sa commandes est toujours là; il
porusuit donc vers le paiement et cette fois, y accède directement.

Arpès avoir régler son achat, il est redirigé vers son compte. Où il est
invité à saisir le reste de ces coordonnées. Dans la partie intitulée
``Mes réservations'', il constate que sa réservation est présente.

Un lien lui permet de visualiser et d'imprimer son billet. Après l'avoir
imprimé, il se déconnecte et quitte le site.
